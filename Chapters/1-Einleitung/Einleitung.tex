%=========================================
% 	   Einleitung     		 =
%=========================================
\chapter{Einleitung}

\section{Motivation}
Ob der Film-Verleih, das Taxi-Geschäft oder der Buchhandel. All diese Wirtschafts-Zweige wurden im letzten Jahrzehnt vollständig transformiert.  Neue Geschäftsmodelle haben alte Denkweisen abgelöst und meist vollständig ersetzt. Dies ist nicht zuletzt auf die Fortschritte in der Software-Entwicklung zurück zu führen. Das diese für neue Geschäftsmodelle von Bedeutung sein wird, bemerkte Watts S. Humphrey bereits im Jahr 2002. In seinem Buch Winning with Software: An Executive Strategy scheibt er: 
\begin{quote}
''Current trends suggest that, regardless of the industry you are in, your future products will use more software and be more complex than those of today.''\cite{humphrey_why_2002}
\end{quote}
Scholl et Al. nennt diese Firmen ''born-in-the-cloud companies, [...]'' \cite{scholl_cloud_2019}.\\
Der Begriff \textit{Cloud} stammt dabei, von dem Begriff \textit{Cloud Computing} ab. Die NIST definiert Cloud Computing als ''[...] on-demand network access to a shared pool of configurable computing resources [...] that can be rapidly provisioned and released with minimal management [...]'' \cite{mell_nist_2011}.\\
In der Praxis bedeutet dies, dass die benötigten Server-Ressourcen nicht selbst verwaltet werden. Man kauft Rechenleistung ein.\\
Dies ermöglicht Firmen den Aspekt der Server-Verwaltung zu extrahieren. Der Fokus kann daher auf das eigene Geschäft und weniger auf die konkrete Infrastruktur gelegt werden. Rechenleistung wird in der Menge eingekauft, wie sie zu einem bestimmten Zeitpunkt benötigt wird.\\
Hierdurch ist es Firmen möglich ihre Angebote flexibel an wachsende Kunden-zahlen anzupassen. Ferner ist es möglich kurzfristig auf Anforderungen des Marktes zu reagieren. Damit sind diese Unternehmen flexibler, als jene die sie ersetzt haben. Applikationen, welche diese Geschäftsmodelle unterstützen und die Möglichkeiten der Cloud ausschöpfen, werden als sogenannte \textit{Cloud-Native Applikationen} bezeichnet \cite{scholl_cloud_2019}.  Sicherheit, Skalierbarkeit und Erweiterbarkeit sind nur ein Teil der Anforderungen, welche bei der Entwicklung einer Cloud-Native Architektur betrachtet werden müssen. 

\section{Struktur der Arbeit}
Diese Arbeit ist wie Folgt strukturiert. Zunächst wird im ersten Kapitel der Begriff Cloud-Computing weiter vertieft. Dabei werden Anforderungen an eine Cloud-Architektur erarbeitet und gezeigt wie Cloud-Native Technologien diese lösen können. Kapitel 2 analysiert die Cloud-Native Architektur anhand ihrer Vor -und Nachteile. In Kapitel 3 wird auf konkrete Technologien für die Umsetzung einer Cloud-Native Architektur eingegangen. Schließlich wird in Kapitel 4 eine Cloud-Native Fallstudie vorgestellt und evaluiert.  

