%=========================================
% 	   Einleitung     		 =
%=========================================
\chapter{Einleitung}

\section{Motivation}
Im Informationszeitalter werden Informationen meist über alles andere gestellt. Wir bleiben allerdings physisch und unsere Umgebung ebenso. Um Computer in diesen physischen Umgebungen zu nutzen, sind viel zu oft noch Informationen von Menschen nötig. Menschen haben allerdings eine limitierte Aufmerksamkeitsspanne, Zeit und Genauigkeit. Computer sind von Informationen abhängig, die von Menschen erstellt, aufgenommen oder eingetippt werden. Menschen brauchen Zeit um Werte aufzunehmen und machen Fehler dabei. Dies führt dazu, dass Computer ebenso fehlerhafte Daten produzieren oder Informationen erst zu spät zur Verfügung stehen. Oft führt 
Computer und Sensoren haben keine schwankende Aufmerksamkeitsspanne und werden nicht müde. Eine Lösung wäre folglich eine Kommunikation zwischen der physischen Welt und der digitalen Welt, ohne den Mensch als Informations Brücke. Dies ist das Ziel der \textit{Internet of Things}. Eine einheitliche Definition des Begriffs Internet of Things existiert nicht. Denn es gibt unterschiedliche Auffassungen davon, was zu den Internet of Things gezählt wird. Maschinelle Sensoren im Bereich der Industrie 4.0 gelten als selbstverständlich, aber man könnte auch sogenannte RFID Tags oder sogar Smartphones hinzuzählen. Ich werde mich hier auf die Definition von Forrest Stroud beziehen: 
\begin{quote}
„The IoT refers to the ever-growing network of physical objects that feature an IP address for the internet connectivity and the communication that occurs between these objects and other internet-enabled devices and systems.“ 
\end{quote}
Stroud bezieht sich demnach auf physische Objekte, welche eine IP-Adresse zur Kommunikation vorweisen. Diese reichen von Geräten der Infrastruktur, wie Ampelanlagen und 

\section{Aufgabenstellung und Zielsetzung}
In meiner Arbeit werde ich Xamarin, React Native, Ionic und Flutter miteinander vergleichen und eine passende Lösung für die Problemstellung des IoT-Testfelds der HTWSaar auswählen. Anschließend wird mit dieser ein Frontend für das SmartCity-Testfeld konzipiert und umgesetzt. Dabei ist das Ziel eine Struktur zu entwickeln, mit welcher eine einfache Erweiterbarkeit und Modifikation erreicht wird. Konkret soll es möglich sein, weitere Use Cases einzubinden. Dadurch wird die Applikation zukunftssicher und flexibel. Zudem kann das SmartCity Testfeld erweitert werden, ohne das ein weiteres Frontend entworfen werden muss. Die erarbeiteten Konzepte sollen anhand einer Proof of Concept Implementierung getestet werden. Dabei wird zunächst eine Frontend-Implementierung des Projekts \textit{Smart-Garbage} erarbeitet.

\section{Struktur der Arbeit}
Diese Arbeit ist wie Folgt strukturiert. Zunächst wird in Kapitel eine Basis an Grundwissen gesammelt. Hierzu zählen eine Betrachtung der verschiedenen 


