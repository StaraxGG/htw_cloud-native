%=========================================
% 	   Einleitung     		 =
%=========================================
\chapter{Einleitung}

\section{Motivation}
Ob Film-Verleih, Taxi-Geschäft oder Buchhandel: All diese Wirtschaftszweige wurden im letzten Jahrzehnt vollständig transformiert. Dies ist nicht zuletzt auf die Fortschritte in der Software-Entwicklung zurück zu führen. Dass diese für neue Geschäftsmodelle von Bedeutung ist, bemerkte Watts S. Humphrey bereits im Jahr 2002. In seinem Buch ''Winning with Software: An Executive Strategy'' scheibt er: 
\begin{quote}
''Current trends suggest that, regardless of the industry you are in, your future products will use more software and be more complex than those of today.'' \cite{humphrey_why_2002}
\end{quote}
Scholl et Al. bezeichnen solche Firmen als ''Born-In-The-Cloud''\cite{scholl_cloud_2019}.\\
Der Begriff \textit{Cloud} stammt dabei von dem Begriff \textit{Cloud Computing} ab. Die National Institute of Standards and Technology definiert Cloud Computing als ''[...] on-demand network access to a shared pool of configurable computing resources [...] that can be rapidly provisioned and released with minimal management [...]''\cite{mell_nist_2011}.\\
In der Praxis bedeutet dies, dass die benötigten Server-Ressourcen nicht von Firmen selbst verwaltet werden müssen. \\
Dies ermöglicht es Firmen den Aspekt der Server-Verwaltung an Dritte auszulagern. Der Fokus kann daher auf das eigene Geschäft und weniger auf die konkrete Infrastruktur gelegt werden. Rechenleistung wird dabei konkret in einem proportionalen Verhältnis zu den aktuellen Bedürfnissen eingekauft.\\
Damit ist es Firmen möglich, ihre Angebote flexibel an wachsende Kunden-zahlen anzupassen. Applikationen, welche diese Geschäftsmodelle nutzen und die Möglichkeiten der Cloud ausschöpfen, werden als \textit{Cloud-Native Applikationen} bezeichnet \cite{scholl_cloud_2019}.  Sicherheit, Skalierbarkeit und Erweiterbarkeit sind nur ein Teil der Anforderungen, welche bei der Entwicklung einer Cloud-Native Architektur betrachtet werden müssen. 

\section{Struktur der Arbeit}
Diese Arbeit ist wie folgt strukturiert. Zunächst wird im ersten Kapitel der Begriff Cloud-Computing näher erläutert. Dabei werden Anforderungen an eine Cloud-Architektur erarbeitet. Weiterhin wird gezeigt wie Cloud-Native Technologien diese lösen können. Kapitel zwei analysiert die Cloud-Native Architektur anhand ihrer Vor -und Nachteile. In Kapitel drei wird auf konkrete Technologien für die Umsetzung einer Cloud-Native Architektur eingegangen. Schließlich wird in Kapitel vier eine Cloud-Native Fallstudie vorgestellt und evaluiert.  

