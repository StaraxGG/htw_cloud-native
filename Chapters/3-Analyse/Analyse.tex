%=========================================
% 	   Analyse     		 =
%=========================================
\chapter{Analyse}
\label{ch:analyse}

\section{Differenzierung}

Dieses Kapitel vergleicht die Prinzipien, Eigenschaften und Umsetzungskriterien der Cloud-Native-Architektur mit denen anderer Architekturen und stell deren Unterschiede dar.

\subsection{Monolithen}

Ein grundlegender Unterschied zwischen einer monolitischen Softwarearchitektur und der einer Cloud-Native-Architektur findet sich in der Zusammensetzung ihrer Komponenten. Ein sogenannter \textit{Monolith} ist ein einheitliches Softwaresystem, welches die gesamte Programmlogik, Datenverwaltung und die Benutzeroberfläche in einer abgeschlossenen, ausführbaren Datei verbindet. Die in ihr enthaltenen Komponenten sind oft eng miteinander verknüpft. Diese Zusammensetzung bietet einen in den Anfangsphasen der Entwicklung simplen Implemtierungsprozess, da die Entwickler während der Umsetzung eines Prototyps unmittelbar Änderungen an den verschiedenen Komponenten umsetzen können. In den Anfängen eines Projektes ist es zudem simpel, Änderung zu veröffentlichen. Dazu muss lediglich eine neue Version der gebündelten Datei des Monoliths an seine Nutzer verteilt werden. Bei zunehmendem Wachstun des Monoliths kristallisieren sich jedoch einige Nachteile heraus. Wächst die Codebasis des Monoliths, erhöht sich die Wahrscheinlichkeit von abnehmender modulatrität seiner Bestandteile. Wenn sie stark miteinander verwoben sind erhöht sich die Komplexität bei notwendigen Veränderungen oder der Einführung neuer Funktionen. Gleichzeitig wä
- Datenverwaltung
- Grundlegende Architektur


\subsection{Serverless}

\subsection{Cloud-Agnostisch}

\section{Evaluation}

\subsection{Vorteile}

\section{Nachteile}

\section{Zusammenfassung}