%=========================================
% 	   Grundlagen     		 =
%=========================================
\chapter{Grundlagen}
\label{ch:grundlagen}
Um die Cloud Native Architektur zu betrachten, gilt es erst einige Grundlagen zu erarbeiten. In diesem Kapitel wird zunächst der Begriff Cloud abgegrenzt. Dabei wird auch der unterschied von \textit{Public} und \textit{Private Cloud} betrachtet. Daraus wird anschließend eine Liste von Anforderungen an eine Cloud-Architektur formuliert. Aufbauend auf dieser wird abschließend die Cloud Native Architektur definiert und deren Eigenschaften betrachtet. 

\section{Die Cloud}
Der Begriff Cloud ist heutzutage sowohl in Fachliteratur, als auch in it-fernen Bereichen vorzufinden. Dabei wird der Begriff \textit{Cloud} als Oberbegriff für neue Technologien und Produkte verwendet.\\ 
Dafür gilt es zunächst zu ermitteln, in welcher Umgebung Cloud-Applikationen betrieben werden.\\\\
Die NIST definiert \textit{Cloud Computing}, als ein Model für allgegenwärtigen Zugriff auf eine geteilte Menge von Rechenleistung. Diese kann durch ein Service-Provider schnell bereitgestellt oder freigegeben werden. Der Kunde kann demnach flexibel Rechenleistung dazu kaufen, oder die bereitgestellte Rechenleistung verringern. Hierfür ist keine menschliche Kommunikation von Nöten. Deshalb wird dies von der NIST als \textit{On-demand self-service} bezeichnet.\\
Um dies zu ermöglichen nutzt der Service-Provider das sogenannte \textit{Resource-Pooling}. Hierfür teilt der Service-Provider dem Kunden automatisiert Rechenleistung zu. Der Kunde hat hierbei nur begrenzt Einfluss auf den Standort oder die genaue Maschine, welche die Leistung zur Verfügung stellt.\cite{mell_nist_2011}\\
Der Kunde bezahlt folglich Rechenleistung, und nicht einzelne Maschinen. Die Cloud kann demnach als Pradigma für das Hosten und Bereitstellen von Services über das Internet gesehen werden \cite{avram_advantages_2014}.

\subsection{Service Modelle}
Verschiedene Geschäftsmodelle des Cloud-Computings setzen unterschiedliche Grade der Abstraktion um. Hierbei wird zwischen den Service Modellen \textit{Infrastructure as a Service}, \textit{Plattform as a Service} und \textit{Software as a Service} unterschieden. Im Nachfolgenden werden diese Modelle voneinander abgegrenzt.
\subsubsection{Infrastructure as a Service}
Das sogenannte\textit{ Infrastructure as a Service (IaaS)}\cite{mell_nist_2011} stellt Speicher -und Rechenleistung zur Verfügung. Der Kunde muss also nicht selbst physische Hardware betreiben. IaaS kann als Grundschicht unter allen anderen Service Modellen gesehen werden.\\
Mittels Virtualisierung teilt der Service-Provider die physische Hardware auf und teilt sie den Kunden zu. Der  Kunde hat also kein Einfluss auf die zugrundeliegende Infrastruktur. Allerdings kann er Aspekte wie Speicher, Betriebssystem, Middle-Ware und die Anwendung selbst konfigurieren \cite{dimpi_rani_rajiv_kumar_ranjan_comparative_2014}.\\
Auch ein begrenzter Zugriff auf Netzwerkomponenten, wie zum Beispiel die Firewall, kann möglich sein. IaaS stellt demnach die selben Funktionen wie ein traditionelles  Rechen-Center zur Verfügung. Der Kunde muss allerdings kein solches Instandhalten und kann flexibel die Rechenleistung erhöhen oder verringern. \\
Abgerechnet wird in der Regel entsprechend der Nutzungsdauer. In einigen Fällen übernimmt der Service-Provider auch Aufgaben wie die Systemwartung, Datensicherung und das Notfall-Management.\\
Unter anderem sind Microsoft, IBM und Amazon Anbieter von IaaS Angeboten. \cite{simon_lohmann_iaas_nodate} \cite{mell_nist_2011}
\subsubsection{Plattform as a Service}
Das \textit{Plattform as a Service (PaaS)}\cite{mell_nist_2011} Model abstrahiert gegenüber IaaS weitere Aspekte der Cloud. Weiterhin muss der Kunde keine eigene Rechenzentren verwalten und erhält durch den Service Provider flexibel Rechenleistung.\\
Allerdings werden nun auch Aspekte wie das Betriebssystem, die Speicherverwaltung mit Datenbanken und das Netzwerk, durch den Service Provider verwaltet. Der Kunde erhält eine Entwicklungsumgebung in der Cloud, über welche der die Anwendung bereitstellen kann. \\
PaaS ist dafür konzipiert den kompletten Lebenszyklus einer Anwendung zu ermöglichen. Hierzu zählen das bauen, testen, veröffentlichen, verwalten und aktualisieren einer Anwendung \cite{dimpi_rani_rajiv_kumar_ranjan_comparative_2014}. Auf einer höheren Abstraktion kommen neben Entwicklungswerkzeugen, Programmiersprachen, Bibliotheken und Datenbanken auch Container-Techniken dazu.\\
Zu den wichtigsten PaaS Anbietern gehören Amazon, IBM und Microsoft. \cite{simon_lohmann_platform_nodate}\cite{sowmya_layers_2014} \cite{mell_nist_2011} 
\subsubsection{Software as a Service}
\textit{Software as a Service (SaaS)} ist die am meisten abstrahierte Form des Cloud-Computings. Kunden können über das Internet auf Angebote zugreifen, welche von einem Service-Provider zur Verfügung gestellt werden. Die Applikation wird durch den Service-Provider verwaltet, betrieben und aktualisiert. \\
Der Nutzer verwaltet keinen Aspekte der zugrundeliegenden Infrastruktur. Einzig eine konfigurieren der Anwendung selbst kann möglich sein.\\
Typische SaaS-Applikationen im Business Bereich sind \textit{Google G Suite} \cite{noauthor_google_nodate}  und \textit{Microsoft Office 365} \cite{noauthor_office_nodate}. \\
SaaS-Provider rechnen Anwendungen in der Regen anhand bestimmter Parameter, wie die Anzahl der Nutzer, ab. Auch bei SaaS-Angeboten können Kunden bei Bedarf einzelne Dienste oder Funktionen stärker in Anspruch nehmen. \cite{wolfgang_herrmann_saas_nodate} \cite{sowmya_layers_2014} \cite{mell_nist_2011}

\subsection{Deployment Modelle}
Neben dem Grad der Abstraktion stellt auch der Ort der Hardware eine Rolle. Wird diese lokal von dem eigenen Unternehmen verwaltet? Oder ist eine externe Firma dafür verantwortlich? Hierfür existieren einige Modell, welche im Folgenden voneinander differenziert werden:
\begin{description}
    \item [on premise] \hfill \\ Auch als on-premise Cloud oder private Cloud bezeichnet \cite{dimpi_rani_rajiv_kumar_ranjan_comparative_2014}. Die Infrastruktur wird exklusiv von einer Organisation genutzt \cite{mell_nist_2011}. Die Verwaltung der Infrastruktur kann ebenfalls durch diese Organisation erfolgen. Alternativ kann eine externe Firma damit beauftragt werden. Die benötigte Hardware befindet sich entweder in-house (\textit{on-premise}) oder oder wird \textit{off-premise} extern bereitgestellt \cite{zwicker_saas_nodate}.
    \item [public Cloud] \hfill \\ Die Infrastruktur wird durch einen Dritt-Anbieter über das Internet zur Verfügung gestellt. Die gesamte Hardware ist im Besitzt des Service-Providers. Die selbe Hardware, Speicher und Netzwerk-Geräte des Service-Providers werden mit anderen Kunden geteilt. Beispiele hierfür wären Microsoft Azure oder Amazon Web Services. \cite{microsoft_public_nodate} \cite{mell_nist_2011}
    \item [hybrid Cloud] \hfill \\ Hybrid Clouds bestehen aus mehreren getrennten Cloud-Infrastrukturen. Daten und Applikationen einer hybrid Cloud können zwischen private Cloud und public Cloud Infrastrukturen wechseln. \cite{mell_nist_2011}
\end{description}

\subsection{Vorteile und Möglichkeiten der Cloud}
Jedes der oben genannten Deployment Modelle und Service Modelle hat eine Menge von Eigenschaften gemeinsam. Avram et al. nennt folgende Eigenschaften: (i) pay-per-use (kein vortlaufenden Verpflichtungen), elastic capacity and the illusion of infinite resources; (iii)self-service-interface; und (iv) resources that are abstracted or virtualized \cite{avram_advantages_2014}. Aus diesen folgend direkt einige Vorteile des Cloud-Computings.
\begin{itemize}
  \item Durch das pay-per-use Modell muss der Kunde nur für die Leistung zahlen, die er tatsächlich benötigt. Dies reduziert Hardware-Kosten, wie auch die Kosten für Verwaltung, Software-Updates und den Strom. Ferner wird weniger Personal für die Verwaltung der IT benötigt. Dies ermöglicht auch kleineren Firmen rechen-aufwändige Analysen zu bewältigen. Letztere waren zuvor nur größeren Firmen vorbehalten. Diese umfangreichen Berechnungen benötigen meist viel Leistung, für einen kurzen Zeitraum. Durch das self-service-interface (iii) und die dynamische Kapazität von Leistung (iv) mit pay-per-use (i) ist dies möglich. Avram et al \cite{avram_advantages_2014} sieht hier zusätzlich eine Möglichkeit für Dritte-Welt-Länder zu dem Westen aufzuschließen. Länder welche traditionell nicht die benötigen Ressourcen gehabt hätten, können nun diese durch Cloud-Computing beziehen.  
  \item Durch das Cloud-Computing können neue Business-Modell umgesetzt werden. Die Cloud ermöglicht einen direkten Zugriff auf Rechenleistung, ohne eine Vorherige Kapital-Anlage des Kunden \cite{noauthor_premise_2020}. Es muss nicht zuvor Geld in ein Rechenzentrum oder zusätzliche IT-Experten investiert werden. Die Zeit bis ein neues Produkt gewinnbringend an den Markt gebracht werden kann, ist demnach deutlich gesunken. Viele Internet-Startups konnten dadurch gegründet und zum Erfolg geführt werden \cite{avram_advantages_2014}. 
  \item Cloud-Computing erleichtert des Firmen, ihre Services zu skalieren. Wächst die Zahl der Nutzer, kann Rechenleistung dazu gekauft werden. Die Kosten steigen daher dynamisch mit der Reichweite des Unternehmens. Auch eine Reaktion auf ein kurzfristigen Anstieg oder Abfall der Nutzer-Anfragen ist möglich. Eines der Ziele des Cloud-Computings besteht darin, Ressourcen mithilfe von Software-APIs je nach Client-Auslastung bei minimaler Interaktion mit dem Service-Provider dynamisch zu vergrößern oder zu verkleinern \cite{avram_advantages_2014}. Dies ist möglich, da die Rechen-Ressourcen durch Software verwaltet werden. Für eine dynamische Skalierung, wird zusätzlich eine passende Software-Architektur benötigt. Hierauf wird in einem späteren Kapitel eingegangen.
  \item Hier ist außerdem der Gedanke aufzugreifen, dass eine Fokussierung auf das Geschäftsmodell möglich wird. Firmen können sich auf ihr eigentliches Geschäftsmodell konzentrieren. Zeit und Geld können in das Geschäftsmodell und nicht in die IT-Infrastruktur investiert werden.
\end{itemize}

\section{Anforderungen}
Nachdem wir im letzten Abschnitt den Begriff Cloud-Computing erläutert haben, gilt es nun Cloud-Applikationen zu betrachten. Die meisten der zuvor genannten Vorteile des Cloud-Computings können nur durch eine passende Software-Architektur vollkommen ausgeschöpft werden. Beim Entwurf der Architektur einer Cloud-Applikation müssen einige Anforderungen beachtet werden. Folgende Anforderungen haben sich dabei herauskristallisiert:
\begin{itemize}
    \item Cloud-Applikationen operieren meist global. Dies heißt nicht nur, dass der Dient stets über das Internet erreichbar ist. Bei der Betrachtung von globalem Zugriff muss berücksichtigt werden, dass der Dienst auch in lokalen Rechenzentren dupliziert werden muss \cite{gannon_cloud-native_2017}. Nur so kann eine geringe Latenz garantiert werden. Dabei führt die Integrität von Daten häufig zu Problem. Diese müssen stets dem Nutzer zur Verfügung stehen. Dem Nutzer darf nicht ersichtlich sein, dass seine Daten an mehreren Orten abgespeichert sind.  
    \item Cloud-Applikationen müssen vielen tausend von gleichzeitigen Nutzern operieren können. Neben einer vertikalen Skalierung (mehr und bessere Hardware) wird demnach auch eine horizontale Skalierung nötig \cite{gannon_cloud-native_2017}. Diese wird meist durch Parallelisierung umgesetzt. Eine horizontale Skalierung führt erneut zu einem Fokus auf die Konsistenz und Synchronisation des Systems. 
    \item Die Architektur einer Cloud-Applikationen muss mit der Annahme entwickelt werden, dass die Hardware nicht konstant ist und Fehler vorkommen werden. Gannon et Al. \cite{gannon_cloud-native_2017} sieht Probleme bei Applikationen, welche nicht für die Cloud entwickelt wurden. Diese nehmen an, dass die Hardware und das Betriebssystem konstant sind. Der kleinste Fehler im Rechenzentrum oder des Netzwerks führen hier bereits zu Fehlern. 
    \item Cloud-Applikationen müssen die Anforderungen an Verteilte-Systeme beachten \cite{scholl_cloud_2019}. Als Beispiel kann das sogenannte CAP-Theorem erwähnt werden \cite{julianbrowne_brewers_nodate}. Ohne eine Optimierung für den Ausfall einzelner (Teil-)Systeme ist die Anforderung \textit{Partition Tolerance} des CAP-Theorem nicht erfüllt. Es stellt sich also heraus, dass verteilte Systeme eine Reihe von Anforderungen mit sich bringen. Die Architektur muss demnach damit umgehen, dass sich Dienste nicht auf der selben Maschine befinden. Es wird folglich mit einem Netzwerk von Maschinen gearbeitet. Es wird die Welt der Verteilten Systeme betreten. Diese bringt eine Reihe von eigenen Anforderungen mit sich. Oft werden hier Falsch-Annahmen gemacht. Scholl et Al. \cite{scholl_cloud_2019} nennt diese Annahmen "Fallacies of Distributed Systems". Dazu gehören Aussagen wie unter Anderem "the Network is reliable", "Latency is zero" oder "There is infinite bandwith". All diese Aspekte führen zu eigenen Anforderungen, welche die Architektur beachten muss.
    \item Die meisten Cloud-Applikationen werden für den Dauerbetrieb entwickelt. Ein zugriff der Nutzer soll stets möglich sein. Hier ist außerdem der Gedanke aufzugreifen, dass Firmen  Verträge mit dem Kunden haben können. Diese beschränken eine mögliche Downtime auf wenige Stunden im Jahr. Ein Einspielen von Updates darf demzufolge nicht zu einer Downtime für den Nutzer führen. Zudem müssen diese Updates stets getestet werden. Auch Fehler im System oder Abstürze einzelner Knoten dürfen nicht das gesamte System, oder einzelne Funktionen dessen, lahmlegen. Ein umfangreiche Überwachung des Systems und das vorhalten von Ersatzkonten sind nur einige der wichtigen Operationen. 
    
\end{itemize}