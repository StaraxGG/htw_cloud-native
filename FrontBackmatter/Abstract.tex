%*******************************************************
% Abstract
%*******************************************************
\pdfbookmark[0]{Zusammenfassung}{Zusammenfassung}
\chapter*{Zusammenfassung}
Weniger als 7\% der Entwickler haben Kenntnisse in den Programmiersprachen Swift, Objective-C und Kotlin, obwohl diese für die Entwicklung von mobilen Applikation benötigt werden \cite{stachoverflow_stack_2019}. Darüber hinaus müssen mehrere Plattformen gepflegt und mit neuen Inhalten versorgt werden. Dies führt zu erhöhten Kosten und gesteigertem Zeitaufwand. Gerade Studenten können während eines Projektes meist nur für eine mobile Plattform entwickeln.\\
Cross-Plattform-Lösungen ermöglichen eine gleichzeitige Entwicklung für	 mehrere Plattformen. Daher wachsen Frameworks wie Google's Flutter, React Native, Ionic und \mbox{Xamarin} in ihrer Beliebtheit. Ebenso wachsen IoT-Konzepte in ihrer Relevanz. Sogenannte SmartCitys setzen diese ein um Abläufe innerhalb von Städten zu automatisieren und optimieren. Die HTWSaar verdeutlicht diese Konzepte anhand einer Miniaturstadt.\\
In dieser Arbeit wird eine plattformunabhängige und skalierbare Benutzerschnittstelle für diese Miniaturstadt konzipiert und umgesetzt.  Dabei werden zunächst die verschiedenen Cross-Plattform-Frameworks anhand gegebener Anforderungen evaluiert um sicher zu stellen, dass die optimale Lösung gewählt wird. Anschließend wird mithilfe von Flutter die Benutzerschnittstelle konzipiert und umgesetzt. Das Bloc-Architekturmuster wird dabei genutzt um die Anwendung modular und skalierbar zu gestalten. Durch Flutter ist die Applikation auf Android und iOS lauffähig. Eine anschließende Evaluation stellte keine Unterschiede anhand von Leistung, Aussehen und Verhalten zu nativen Applikationen fest. Ferner ermöglicht Flutter mit modernsten Entwicklungs-Tools eine effiziente Entwicklung.\\
Arbeiten von Heitkötter et al. \cite{heitkotter_evaluating_2013} und Shah et al. \cite{shah_analysis_2019} betrachteten bereits verschiedene Cross-Plattform Frameworks anhand von abstrakten Kriterien. Das Ziel dieser Arbeit ist die Evaluation von Cross-Plattform Lösungen anhand eines konkreten Anwendungsfalls mit Hinblick auf die Internet of Things. Dadurch können neue Aspekte, wie die Anbindung von Cross-Plattform Lösungen an IoT-Systeme, betrachtet werden. Darüber hinaus wird eine konkrete Architektur für die Entwicklung mit Flutter vorgestellt. Letztendlich kann diese Arbeit als Beispiel für die Entwicklung einer skalierbaren Flutter-Anwendung dienen. 





%Letztendlich stellt diese Arbeit eine skalierbare Architektur für Flutter vor, welche als Wegweiser für die Entwicklung einer skalierbaren Flutter-Anwendung dienen kann.



%Darüber hinaus wird eine skalierbare Architektur für die Entwicklung mit Flutter betrachten

%. Diese Architektur kann als Referenz für ähnliche Projekte dienen und anderen Entwicklern helfen. 